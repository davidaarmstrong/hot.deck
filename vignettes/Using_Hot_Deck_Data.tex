\documentclass[12pt]{article}\usepackage[]{graphicx}\usepackage[]{color}
% maxwidth is the original width if it is less than linewidth
% otherwise use linewidth (to make sure the graphics do not exceed the margin)
\makeatletter
\def\maxwidth{ %
  \ifdim\Gin@nat@width>\linewidth
    \linewidth
  \else
    \Gin@nat@width
  \fi
}
\makeatother

\definecolor{fgcolor}{rgb}{0.345, 0.345, 0.345}
\newcommand{\hlnum}[1]{\textcolor[rgb]{0.686,0.059,0.569}{#1}}%
\newcommand{\hlstr}[1]{\textcolor[rgb]{0.192,0.494,0.8}{#1}}%
\newcommand{\hlcom}[1]{\textcolor[rgb]{0.678,0.584,0.686}{\textit{#1}}}%
\newcommand{\hlopt}[1]{\textcolor[rgb]{0,0,0}{#1}}%
\newcommand{\hlstd}[1]{\textcolor[rgb]{0.345,0.345,0.345}{#1}}%
\newcommand{\hlkwa}[1]{\textcolor[rgb]{0.161,0.373,0.58}{\textbf{#1}}}%
\newcommand{\hlkwb}[1]{\textcolor[rgb]{0.69,0.353,0.396}{#1}}%
\newcommand{\hlkwc}[1]{\textcolor[rgb]{0.333,0.667,0.333}{#1}}%
\newcommand{\hlkwd}[1]{\textcolor[rgb]{0.737,0.353,0.396}{\textbf{#1}}}%
\let\hlipl\hlkwb

\usepackage{framed}
\makeatletter
\newenvironment{kframe}{%
 \def\at@end@of@kframe{}%
 \ifinner\ifhmode%
  \def\at@end@of@kframe{\end{minipage}}%
  \begin{minipage}{\columnwidth}%
 \fi\fi%
 \def\FrameCommand##1{\hskip\@totalleftmargin \hskip-\fboxsep
 \colorbox{shadecolor}{##1}\hskip-\fboxsep
     % There is no \\@totalrightmargin, so:
     \hskip-\linewidth \hskip-\@totalleftmargin \hskip\columnwidth}%
 \MakeFramed {\advance\hsize-\width
   \@totalleftmargin\z@ \linewidth\hsize
   \@setminipage}}%
 {\par\unskip\endMakeFramed%
 \at@end@of@kframe}
\makeatother

\definecolor{shadecolor}{rgb}{.97, .97, .97}
\definecolor{messagecolor}{rgb}{0, 0, 0}
\definecolor{warningcolor}{rgb}{1, 0, 1}
\definecolor{errorcolor}{rgb}{1, 0, 0}
\newenvironment{knitrout}{}{} % an empty environment to be redefined in TeX

\usepackage{alltt}
\usepackage{amsfonts, amsmath, amssymb, bm} %Math fonts and symbols
\usepackage{dcolumn, multirow} % decimal-aligned columns, multi-row cells
\usepackage{graphicx, subfigure, float} % graphics commands 
\usepackage[margin=1in]{geometry} % sets page layout
\usepackage{setspace}% allows toggling of double/single-spacing
\usepackage{verbatim}% defines environment for un-evaluated code
\usepackage{rotating}% defines commands for rotating text and floats
\usepackage{natbib}% defines citation commands and environments.
\usepackage{url}% allows including urls with \url{} field.
\singlespace % set document spacing to single
\bibpunct[, ]{(}{)}{,}{a}{}{,} % sets the punctuation of the bibliography entires.
\newcolumntype{d}[1]{D{.}{.}{#1}} % defines a decimal-aligned column

\author{Skyler Cranmer\\Ohio State University \and Jeff Gill\\Washington University St. Louis \and Natalie Jackson\\The Huffington Post \and Andreas Murr\\University of Oxford \and David A. Armstrong II\\University of Wisconsin-Milwaukee}

\title{Using Multiple Hot Deck Data Sets for Inference}
%\VignetteIndexEntry{Using Multiple Hot Deck Data for Inference}
\IfFileExists{upquote.sty}{\usepackage{upquote}}{}
\begin{document}

\maketitle





This document will walk you through some of the methods you could use to generate pooled model results that account for both sampling variability and across imputation variability.  The package \verb"hot.deck" does not come with a set of functions to do inference, so we will show you how you could use the data generated by \verb"hot.deck" in combination with \verb"glm.mids" (and similarly \verb"lm.mids") from the \verb"mice" package, \verb"zelig" from the \verb"Zelig" package and by using \verb"MIcombine" from the \verb"mitools" package on a list of model objects.  

\section{Generating Imputations}

The data we will use come from \citet{Poeetal1999} dealing with democracy and state repression.  First we need to call the \verb"hot.deck" routine on the dataset.  

\begin{scriptsize}
\begin{knitrout}
\definecolor{shadecolor}{rgb}{0.969, 0.969, 0.969}\color{fgcolor}\begin{kframe}
\begin{alltt}
\hlkwd{library}\hlstd{(hot.deck)}
\hlkwd{data}\hlstd{(isq99)}
\hlstd{out} \hlkwb{<-} \hlkwd{hot.deck}\hlstd{(isq99,} \hlkwc{sdCutoff}\hlstd{=}\hlnum{3}\hlstd{,} \hlkwc{IDvars} \hlstd{=} \hlkwd{c}\hlstd{(}\hlstr{"IDORIGIN"}\hlstd{,} \hlstr{"YEAR"}\hlstd{))}
\end{alltt}


{\ttfamily\noindent\color{warningcolor}{\#\# Warning in hot.deck(isq99, sdCutoff = 3, IDvars = c("{}IDORIGIN"{}, "{}YEAR"{})): 52 observations with no observed data.\ \ These observations were removed}}

{\ttfamily\noindent\color{warningcolor}{\#\# Warning in hot.deck(isq99, sdCutoff = 3, IDvars = c("{}IDORIGIN"{}, "{}YEAR"{})): 45 of 4661 imputations with \# donors < 5, consider increasing sdCutoff or using method='p.draw'}}\end{kframe}
\end{knitrout}

\end{scriptsize} 
This shows us that there are still 47 observations with fewer than 5 donors.  Using a different method or further widening the \verb"sdCutoff" parameter may alleviate the problem.  If you want to see the frequency distribution of the number of donors, you could look at: 

\begin{scriptsize}
\begin{knitrout}
\definecolor{shadecolor}{rgb}{0.969, 0.969, 0.969}\color{fgcolor}\begin{kframe}
\begin{alltt}
\hlstd{numdonors} \hlkwb{<-} \hlkwd{sapply}\hlstd{(out}\hlopt{$}\hlstd{donors, length)}
\hlstd{numdonors} \hlkwb{<-} \hlkwd{sapply}\hlstd{(out}\hlopt{$}\hlstd{donors, length)}
\hlstd{numdonors} \hlkwb{<-} \hlkwd{ifelse}\hlstd{(numdonors} \hlopt{>} \hlnum{5}\hlstd{,} \hlnum{6}\hlstd{, numdonors)}
\hlstd{numdonors} \hlkwb{<-} \hlkwd{factor}\hlstd{(numdonors,} \hlkwc{levels}\hlstd{=}\hlnum{1}\hlopt{:}\hlnum{6}\hlstd{,} \hlkwc{labels}\hlstd{=}\hlkwd{c}\hlstd{(}\hlnum{1}\hlopt{:}\hlnum{5}\hlstd{,} \hlstr{">5"}\hlstd{))}
\hlkwd{table}\hlstd{(numdonors)}
\end{alltt}
\begin{verbatim}
## numdonors
##    1    2    3    4    5   >5 
##   18   10   11    6   20 4596
\end{verbatim}
\end{kframe}
\end{knitrout}

\end{scriptsize}
Before running a model, three variables have to be created from those existing.  Generally, if variables are deterministic functions of other variables (e.g., transformations, lags, etc...) it is advisable to impute the constituent variables of the calculations and then do the calculations after the fact.  Here, we need to lag the \verb"AI" variable and create percentage change variables for both population and per-capita GNP.  First, to create the lag of \verb"AI", \verb"PCGNP" and \verb"LPOP".  To do this, we will make a little function.  


\begin{scriptsize}
\begin{knitrout}
\definecolor{shadecolor}{rgb}{0.969, 0.969, 0.969}\color{fgcolor}\begin{kframe}
\begin{alltt}
\hlstd{tscslag} \hlkwb{<-} \hlkwa{function}\hlstd{(}\hlkwc{dat}\hlstd{,} \hlkwc{x}\hlstd{,} \hlkwc{id}\hlstd{,} \hlkwc{time}\hlstd{)\{}
        \hlstd{obs} \hlkwb{<-} \hlkwd{apply}\hlstd{(dat[,} \hlkwd{c}\hlstd{(id, time)],} \hlnum{1}\hlstd{, paste,} \hlkwc{collapse}\hlstd{=}\hlstr{"."}\hlstd{)}
        \hlstd{tm1} \hlkwb{<-} \hlstd{dat[[time]]} \hlopt{-} \hlnum{1}
        \hlstd{lagobs} \hlkwb{<-} \hlkwd{apply}\hlstd{(}\hlkwd{cbind}\hlstd{(dat[[id]], tm1),} \hlnum{1}\hlstd{, paste,} \hlkwc{collapse}\hlstd{=}\hlstr{"."}\hlstd{)}
        \hlstd{lagx} \hlkwb{<-} \hlstd{dat[}\hlkwd{match}\hlstd{(lagobs, obs), x]}
\hlstd{\}}
\hlkwa{for}\hlstd{(i} \hlkwa{in} \hlnum{1}\hlopt{:}\hlkwd{length}\hlstd{(out}\hlopt{$}\hlstd{data))\{}
    \hlstd{out}\hlopt{$}\hlstd{data[[i]]}\hlopt{$}\hlstd{lagAI} \hlkwb{<-} \hlkwd{tscslag}\hlstd{(out}\hlopt{$}\hlstd{data[[i]],} \hlstr{"AI"}\hlstd{,} \hlstr{"IDORIGIN"}\hlstd{,} \hlstr{"YEAR"}\hlstd{)}
    \hlstd{out}\hlopt{$}\hlstd{data[[i]]}\hlopt{$}\hlstd{lagPCGNP} \hlkwb{<-} \hlkwd{tscslag}\hlstd{(out}\hlopt{$}\hlstd{data[[i]],} \hlstr{"PCGNP"}\hlstd{,} \hlstr{"IDORIGIN"}\hlstd{,} \hlstr{"YEAR"}\hlstd{)}
    \hlstd{out}\hlopt{$}\hlstd{data[[i]]}\hlopt{$}\hlstd{lagLPOP} \hlkwb{<-} \hlkwd{tscslag}\hlstd{(out}\hlopt{$}\hlstd{data[[i]],} \hlstr{"LPOP"}\hlstd{,} \hlstr{"IDORIGIN"}\hlstd{,} \hlstr{"YEAR"}\hlstd{)}
\hlstd{\}}
\end{alltt}
\end{kframe}
\end{knitrout}

\end{scriptsize}
Now, we can use the lagged values of \verb"PCGNP" and \verb"LPOP", to create percentage change variables: 

\begin{scriptsize}
\begin{knitrout}
\definecolor{shadecolor}{rgb}{0.969, 0.969, 0.969}\color{fgcolor}\begin{kframe}
\begin{alltt}
\hlkwa{for}\hlstd{(i} \hlkwa{in} \hlnum{1}\hlopt{:}\hlkwd{length}\hlstd{(out}\hlopt{$}\hlstd{data))\{}
    \hlstd{out}\hlopt{$}\hlstd{data[[i]]}\hlopt{$}\hlstd{pctchgPCGNP} \hlkwb{<-} \hlkwd{with}\hlstd{(out}\hlopt{$}\hlstd{data[[i]],} \hlkwd{c}\hlstd{(PCGNP}\hlopt{-}\hlstd{lagPCGNP)}\hlopt{/}\hlstd{lagPCGNP)}
    \hlstd{out}\hlopt{$}\hlstd{data[[i]]}\hlopt{$}\hlstd{pctchgLPOP} \hlkwb{<-} \hlkwd{with}\hlstd{(out}\hlopt{$}\hlstd{data[[i]],} \hlkwd{c}\hlstd{(LPOP}\hlopt{-}\hlstd{lagLPOP)}\hlopt{/}\hlstd{lagLPOP)}
\hlstd{\}}
\end{alltt}
\end{kframe}
\end{knitrout}

\end{scriptsize}
\section{Running Models on Multiple Hot Decking Result}

\subsection{Using Zelig}

In version $\geq 5.0$ of \verb"Zelig", the output from \verb"hot.deck" will have to be converted into a format that looks like Amelia's.  You can do this as follows: 

\begin{knitrout}
\definecolor{shadecolor}{rgb}{0.969, 0.969, 0.969}\color{fgcolor}\begin{kframe}
\begin{alltt}
\hlstd{out} \hlkwb{<-} \hlkwd{hd2amelia}\hlstd{(out)}
\end{alltt}
\end{kframe}
\end{knitrout}

\noindent Then, with the output in the appropriate format, we can use \verb"Zelig" to do the modeling. 



\begin{scriptsize}
\begin{knitrout}
\definecolor{shadecolor}{rgb}{0.969, 0.969, 0.969}\color{fgcolor}\begin{kframe}
\begin{alltt}
\hlkwd{library}\hlstd{(Zelig)}
\hlstd{z} \hlkwb{<-} \hlkwd{zelig}\hlstd{(AI} \hlopt{~} \hlstd{lagAI} \hlopt{+} \hlstd{pctchgPCGNP} \hlopt{+} \hlstd{PCGNP} \hlopt{+} \hlstd{pctchgLPOP} \hlopt{+} \hlstd{LPOP} \hlopt{+} \hlstd{MIL2} \hlopt{+} \hlstd{LEFT} \hlopt{+}
    \hlstd{BRIT} \hlopt{+} \hlstd{POLRT} \hlopt{+} \hlstd{CWARCOW} \hlopt{+} \hlstd{IWARCOW2,} \hlkwc{data}\hlstd{=out,} \hlkwc{model}\hlstd{=}\hlstr{"normal"}\hlstd{,} \hlkwc{cite}\hlstd{=}\hlnum{FALSE}\hlstd{)}
\hlkwd{summary}\hlstd{(z)}
\end{alltt}
\begin{verbatim}
## Model: Combined Imputations 
## 
##              Estimate Std.Error z value Pr(>|z|)
## (Intercept)  5.31e-01  1.34e-01    3.95  7.8e-05
## lagAI        4.68e-01  2.20e-02   21.29  < 2e-16
## pctchgPCGNP  5.44e-03  7.35e-03    0.74   0.4594
## PCGNP       -1.99e-05  3.03e-06   -6.56  5.5e-11
## pctchgLPOP  -5.07e-01  1.08e+00   -0.47   0.6374
## LPOP         7.28e-02  8.09e-03    9.01  < 2e-16
## MIL2         1.16e-01  4.41e-02    2.63   0.0085
## LEFT        -1.41e-01  4.93e-02   -2.85   0.0043
## BRIT        -1.34e-01  3.12e-02   -4.28  1.8e-05
## POLRT       -6.96e-02  1.13e-02   -6.14  8.5e-10
## CWARCOW      6.12e-01  5.58e-02   10.96  < 2e-16
## IWARCOW2     1.68e-01  5.65e-02    2.98   0.0029
## 
## For results from individual imputed datasets, use summary(x, subset = i:j)
## Next step: Use 'setx' method
\end{verbatim}
\end{kframe}
\end{knitrout}

\end{scriptsize}
Note that the summary indicates that the results have been combined across 5 multiply imputed datasets. 

\subsection{Using MIcombine}

You can use the \verb"MIcombine" command from the \verb"mitools" package to generate inferences, too.  Here, you have to produce a list of model estimates and the function will combine across the different results.  

\begin{scriptsize}
\begin{knitrout}
\definecolor{shadecolor}{rgb}{0.969, 0.969, 0.969}\color{fgcolor}\begin{kframe}
\begin{alltt}
\hlcom{# initialize list}
\hlstd{results} \hlkwb{<-} \hlkwd{list}\hlstd{()}
\hlcom{# loop over imputed datasets}
\hlkwa{for}\hlstd{(i} \hlkwa{in} \hlnum{1}\hlopt{:}\hlkwd{length}\hlstd{(out}\hlopt{$}\hlstd{imputations))\{}
    \hlstd{results[[i]]} \hlkwb{<-} \hlkwd{lm}\hlstd{(AI} \hlopt{~} \hlstd{lagAI} \hlopt{+} \hlstd{pctchgPCGNP} \hlopt{+} \hlstd{PCGNP} \hlopt{+} \hlstd{pctchgLPOP} \hlopt{+} \hlstd{LPOP} \hlopt{+} \hlstd{MIL2} \hlopt{+} \hlstd{LEFT} \hlopt{+}
    \hlstd{BRIT} \hlopt{+} \hlstd{POLRT} \hlopt{+} \hlstd{CWARCOW} \hlopt{+} \hlstd{IWARCOW2,} \hlkwc{data}\hlstd{=out}\hlopt{$}\hlstd{imputations[[i]])}
\hlstd{\}}
\hlkwd{summary}\hlstd{(mitools}\hlopt{::}\hlkwd{MIcombine}\hlstd{(results))}
\end{alltt}
\begin{verbatim}
## Multiple imputation results:
##       MIcombine.default(results)
##                   results           se        (lower        upper) missInfo
## (Intercept)  5.306640e-01 1.343256e-01  2.662818e-01  7.950461e-01     12 %
## lagAI        4.684916e-01 2.200209e-02  4.220845e-01  5.148987e-01     54 %
## pctchgPCGNP  5.437420e-03 7.348930e-03 -1.311326e-02  2.398810e-02     90 %
## PCGNP       -1.988956e-05 3.032863e-06 -2.589274e-05 -1.388637e-05     19 %
## pctchgLPOP  -5.073886e-01 1.076611e+00 -2.869658e+00  1.854881e+00     65 %
## LPOP         7.284485e-02 8.087428e-03  5.699280e-02  8.869691e-02      1 %
## MIL2         1.159230e-01 4.407117e-02  2.488961e-02  2.069563e-01     46 %
## LEFT        -1.406836e-01 4.929611e-02 -2.403317e-01 -4.103552e-02     35 %
## BRIT        -1.337468e-01 3.122174e-02 -1.950569e-01 -7.243661e-02      8 %
## POLRT       -6.963110e-02 1.134924e-02 -9.367434e-02 -4.558786e-02     55 %
## CWARCOW      6.119057e-01 5.583945e-02  5.009920e-01  7.228195e-01     23 %
## IWARCOW2     1.680549e-01 5.646443e-02  5.600692e-02  2.801029e-01     22 %
\end{verbatim}
\end{kframe}
\end{knitrout}

\end{scriptsize}

\subsection{Using mids}

The final method for combining results is to convert the data object returned by the \verb"hot.deck" function to an object of class \verb"mids".  This can be done with the \verb"datalist2mids" function from the \verb"miceadds" package. 

\begin{scriptsize}
\begin{knitrout}
\definecolor{shadecolor}{rgb}{0.969, 0.969, 0.969}\color{fgcolor}\begin{kframe}
\begin{alltt}
\hlstd{out.mids} \hlkwb{<-} \hlstd{miceadds}\hlopt{::}\hlkwd{datalist2mids}\hlstd{(out}\hlopt{$}\hlstd{imputations)}
\end{alltt}


{\ttfamily\noindent\color{warningcolor}{\#\# Warning: Number of logged events: 1}}\begin{alltt}
\hlstd{s} \hlkwb{<-} \hlkwd{summary}\hlstd{(mice}\hlopt{::}\hlkwd{pool}\hlstd{(mice}\hlopt{::}\hlkwd{lm.mids}\hlstd{(AI} \hlopt{~} \hlstd{lagAI} \hlopt{+} \hlstd{pctchgPCGNP} \hlopt{+} \hlstd{PCGNP} \hlopt{+} \hlstd{pctchgLPOP} \hlopt{+} \hlstd{LPOP} \hlopt{+} \hlstd{MIL2} \hlopt{+} \hlstd{LEFT} \hlopt{+}
\hlstd{BRIT} \hlopt{+} \hlstd{POLRT} \hlopt{+} \hlstd{CWARCOW} \hlopt{+} \hlstd{IWARCOW2,} \hlkwc{data}\hlstd{=out.mids)))}
\hlkwd{print}\hlstd{(s,} \hlkwc{digits}\hlstd{=}\hlnum{4}\hlstd{)}
\end{alltt}
\begin{verbatim}
##               estimate std.error statistic      df   p.value
## (Intercept)  5.387e-01 1.491e-01    3.6135  50.491 6.956e-04
## lagAI        4.708e-01 1.941e-02   24.2476  41.152 0.000e+00
## pctchgPCGNP  3.573e-03 5.049e-03    0.7076   6.714 5.030e-01
## PCGNP       -1.963e-05 3.791e-06   -5.1778  16.724 7.966e-05
## pctchgLPOP  -8.040e-01 1.377e+00   -0.5840   6.472 5.790e-01
## LPOP         7.150e-02 9.197e-03    7.7742  72.690 3.829e-11
## MIL2         1.209e-01 5.345e-02    2.2619  11.150 4.464e-02
## LEFT        -1.342e-01 4.810e-02   -2.7912  52.087 7.324e-03
## BRIT        -1.369e-01 3.951e-02   -3.4650  22.430 2.156e-03
## POLRT       -6.755e-02 1.090e-02   -6.1990  19.904 4.793e-06
## CWARCOW      6.027e-01 5.755e-02   10.4732  65.585 1.332e-15
## IWARCOW2     1.604e-01 5.362e-02    2.9921 319.263 2.986e-03
\end{verbatim}
\end{kframe}
\end{knitrout}
\end{scriptsize}

\bibliographystyle{apsr}
\bibliography{hot.deck}


\end{document}
